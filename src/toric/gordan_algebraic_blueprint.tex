\documentclass[english]{amsart}

\usepackage{amsmath}
\usepackage{amssymb}

\newcommand{\C}{\mathbb{C}}
\newcommand{\N}{\mathbb{N}}
\newcommand{\Q}{\mathbb{Q}}
\newcommand{\R}{\mathbb{R}}
\newcommand{\Z}{\mathbb{Z}}

\newcommand{\Az}{A_{\geq0}}
\newcommand{\Ap}{A_{>0}}
\newcommand{\ev}{\textrm{deg}_\varphi}


\DeclareMathOperator{\Hom}{Hom}
\DeclareMathOperator{\rk}{rk\,}

\usepackage{amsthm}

\newtheorem{theorem}{Theorem}
\newtheorem{lemma}[theorem]{Lemma}


\begin{document}
\title[]{Blueprint for Gordan's Lemma}

\begin{abstract}
A blueprint for Gordan's Lemma.
\end{abstract}

\maketitle

What we are going for is the following. Let $\Lambda$ be a finite free $\Z$-module, and let $\Lambda^*:=\Hom(\Lambda,\Z)$ be its dual. If $S$ is a subset of $\Lambda^*$ (always finite, in practice, I think) then its \emph{dual $\Z$-cone} $S^\vee$ consists of the $t\in\Lambda$ such that $\langle s,t\rangle\geq0$ for all $s\in S$. Our goal is:

\begin{theorem}
If $S \subset \Lambda^*$ is finite, then $S^\vee \subset \Lambda$ is a finitely-generated additive monoid.
\end{theorem}

\section{Algebraic proof}

We follow the algebraic proof in Wikipedia.

\begin{theorem} \label{t:RMfg}
If $M$ is an additive abelian monoid and $R$ is a nonzero commutative ring, then $M$ is finitely-generated as a monoid iff the monoid algebra $R[M]$ is finitely-generated as an $R$-algebra.
\end{theorem}
\begin{proof} Because $R$ is nonzero, we can think of $M$ as a subset of $R[M]$.
  %DT: even as a submonoid.

  First say $M$ is finitely generated by $S\subseteq M$. The sub-$R$-algebra of $R[M]$ generated by $S$ contains all of $M$, and is an $R$-module so it's all of $R[M]$.

  Conversely, say $R[M]$ is finitely-generated. If $f\in R[M]$ then it has a support, which is a finite subset of~$M$. Now $f$ is in the sub-$R$-module generated by its support, and hence is in the sub-$R$-algebra generated by the support. So if we have finitely many $f$'s which generate $R[M]$ as an $R$-algebra then we can replace each $f$ by its support and get a finite subset of $M$ which generates a subalgebra of $R[M]$ which contains all the $f$'s and hence is all of $R[M]$. We deduce that a finite subset $S$ of $M$ generates $R[M]$ as an $R$-algebra. We claim that $S$ generates $M$ as an additive monoid. This follows because the $R$-algebra generated by $S$ is the $R$-module generated by the monoid generated by $S$, and the monoid generated by $S$ is a subset of $M$.
\end{proof}

We prove Gordan's Lemma by two layers of induction.  First, proceed by induction on the rank of~$\Lambda$.

The result is clear in the case in which the rank of~$\Lambda$ is~$0$: in this case, $\Lambda = 0$ and the empty set generates the unique additive submodule of $\Lambda = 0$.

Assume that the rank of~$\Lambda$ is strictly positive and that, for every free, finitely generated $\Z$-module $\Lambda'$ of rank strictly smaller than the rank of~$\Lambda$, the dual set of every finite subset of $\left(\Lambda'\right)^*$ is a finitely generated additive monoid.

Proceed by induction on (the size of)~$S$.

For~$S$ empty the result is clear: the dual of the empty set is the whole~$\Lambda$ and if $\lambda \subset \Lambda$ generates~$\Lambda$ as a $\Z$-module, then $\lambda \cup \{ -\ell \, : \, \ell \in \lambda \}$ generates~$\Lambda$ as an additive monoid.

For the inductive (in the size of $\#S$) step, it suffices to check that if $S^\vee$ is finitely-generated then so is $(S\cup{\varphi})^\vee$.  We use the equality
\[
(S\cup{\varphi})^\vee = S^\vee \cap \{v \in \Lambda \,:\, \varphi(v)\geq0\,\},
\]
which follows from the definitions.

The result is clear if $\varphi = 0$: in this case $\varphi$ imposes no extra condition on $S^\vee$, the equality
\[
(S \cup \{0\})^\vee = S^\vee
\]
holds, and we know the result for $S^\vee$.

Thus, assume that~$\varphi$ is non-zero.  Set $M=S^\vee$ and write $A=\C[M]$; this is finitely-generated as a $\mathbb{C}$-algebra by Theorem~\ref{t:RMfg}. Define
\[
\ev \colon M\to\Z
\]
by $\ev (v)=\varphi(v)$. Define $A_n$ to be the $\C$-module generated by the $v \in M$ with $\ev(v)=n$; this determines a $\Z$-grading on~$A$. By Theorem~\ref{t:RMfg}, it suffices to prove that the subring $A_{\geq0}:=\oplus_{n\geq0}A_n$ is finitely-generated as a $\mathbb{C}$-algebra.

First note that $A_0=\C[T]$ where $T=\{v \in M \,:\, \deg(v)=0\}$ is a subalgebra, so it suffices to prove that
\begin{itemize}
\item
$A_0$ is a finitely-generated $\C$-algebra, and that
\item
$\Az$ is a finitely-generated $A_0$-algebra.
\end{itemize}

\begin{lemma}
The $\C$-algebra $A_0 = \C[T]$ is finitely generated.
\end{lemma}
\begin{proof}
We use the equivalence of Theorem~\ref{t:RMfg}: it suffices to show that~$T$ is finitely generated as a monoid.  Recall that, by definition,~$T$ is the submonoid of~$\Lambda$ satisfying
\[
T=\{v \in M \,:\, \deg(v) = \varphi (v) = 0\} \subset \ker \varphi .
\]
Since we reduced to the case in which~$\varphi$ is non-zero, we know that $\ker \varphi$ is a free, finitely generated $\Z$-module of rank equal to ${\rk} \Lambda - 1$.

To apply the induction hypothesis, we check that~$T \subset \ker \varphi$ is the dual of a finite subset of $\ker \varphi$.  Observe that the dual of $\ker \varphi$ is the quotient of $\Lambda^*$ by the saturation of the additive subgroup generated by~$\varphi$.  By construction,~$T$ is therefore the dual set of the image of~$S$ under the projection
\[
\Lambda^* \to \left( \Lambda^* / \langle \varphi \rangle \right)^{\textrm{sat}} \simeq \left( \ker \varphi \right)^*.
\]
By the induction step of the first induction (on the number of generators of~$\Lambda$), we know that~$T$ is finitely generated, as needed.
\end{proof}


\noindent
{\textbf{Remark.}}
The saturation can really be avoided, by working more generally, not with the dual of~$\Lambda$, but with a $\Z$-module of linear functionals on~$\Lambda$ that surjects onto the dual of~$\Lambda$.  Alternatively, it can also be avoided by replacing $\varphi$ by $\varphi' \in \Lambda'$, where $\varphi = a \varphi'$, with $a \in \N$ chosen as the largest it can be for such an identity to hold.


\begin{lemma}
The~$A$ be a Noetherian $\Z$-graded ring.  Denote by $\Az = \oplus_{n \geq 0} A_n$ the sub-algebra of~$A$ consisting of the elements of~$A$ of non-negative degree.  The ring $\Az$ is finitely generated as an $A_0$-algebra.
\end{lemma}

\begin{proof}
Let~$I$ be the ideal of~$A$ that is generated by all the homogeneous elements of strictly positive degree.  (Note that, since~$A$ might have elements of negative degree, the ideal~$I$ might contain elements of negative degree as well.)

Since~$A$ is Noetherian, the ideal~$A$ admits a finite generating set: choose one and denote its elements by $f_1, \ldots , f_n$.  We further assume that the chosen generators are
\begin{itemize}
\item
homogeneous (by replacing the non-homogeneous ones by their homogeneous components), and
\item
have strictly positive degree (since the ideal~$I$ is generated by homogeneous elements of strictly positive degrees).
\end{itemize}
Let $N_0 \in \N$ be the maximum of the degrees of the generators $f_1, \ldots , f_n$:
\[
N_0 = \max \{ \deg f_1 , \ldots , \deg f_n \}.
\]
Let $G \subset \Az$ be the subset consisting of $f_1, \ldots , f_n$ together with all the homogeneous elements of degree at most~$N_0$.  We show that~$G$ generates~$\Az$ as an $A_0$-algebra.

More precisely, we show that, for all $n \in \N$, every element $f \in \Az$ of degree at most~$n$ in the $A_0$-algebra $\Az$ is generated by~$G$ as an $A_0$-algebra. (If this is any help, this step is entirely analogous to the proof that the Weak Mordell-Weil Theorem implies the Mordell-Weil Theorem.)

Proceed by induction on~$n$.  For the base case there is nothing to prove, since the result is true if $n \leq N_0$, by definition of~$G$.

Suppose that~$G$ generates every element of~$\Az$ of degree at most~$n$, for some natural number~$n$ satisfying $N_0 \le n$.  Let~$f$ be an element of~$\Az$ of degree $n+1$.  By homogeneity of the ideal, we can assume that~$f$ is homogeneous of degree~$n+1$.

Since $f_1, \ldots, f_n$ generate~$I$, the homogeneous element~$f$ admits a decomposition
\[
f = \sum_{i = 1}^n g_i f_i
\]
with $g_1, \ldots, g_n$ homogeneous elements.  Since the degree of$f$ is~$n+1$ and~$n$ satisfies $N_0 \le n$, the degrees of $g_1, \ldots , g_n$ satisfy
\[
0 \le \deg g_1 < \deg f, \ldots , 0 \le \deg g_1 < \deg f.
\]
By the inductive hypothesis, each one of the elements $g_1, \ldots , g_n$ is in the $\A_0$-algebra generated by~$G$, as stated.

All that is left to show is that, for each natural number~$n$, the homogeneous degree piece~$A_n$ is finitely generated as an~$A_0$-module.

This is again a consequence of Noetherianity of~$A$.  Suppose that $N_1 \subset N_2 \cdots \subset N_i \subset \cdots$ is an increasing chain of $A_0$-submodules of $A_n$, such that $\cup_i N_i = A_n$.  The chain of ideals $N_1 A \subset N_2 A \cdots \subset N_i A \subset \cdots$ stabilizes, since~$A$ is Noetherian: there is an index~$i$ such that, for all $i \le j$, the equality $N_i A = N_j A$ holds.  Intersecting with $A_n$, we find that the sequence $N_1 A \cap A_n \subset N_2 A \cap A_n \cdots \subset N_i A \cap A_n \subset \cdots$ also stabilizes.  Finally, we observe that, for all indices~$i$, the equality $N_i A \cap A_n = N_i$ holds: since all the elements of~$N_i$ are homogeneous of degree~$n$, the only $A$-multiples of the elements of~$N_i$ that have degree~$n$ are the multiples by homogeneous elements of degree~$0$.  Since~$N_i$ is an $A_0$-module, we are done.
\end{proof}
